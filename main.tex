\documentclass{article}[draft]
\usepackage[utf8]{inputenc}

\usepackage{graphicx}
\usepackage{amsmath}
\usepackage{siunitx}
\usepackage{booktabs}

\begin{document}

\begin{center}
  \fontsize{14}{20}{\bf An Optimal Handling Bicycle}
\end{center}

\begin{center}
  \normalsize{\bf{Jason K. Moore, Mont Hubbard,  Ronald A. Hess}}
\end{center}

\begin{center}
  \begin{tabular}{c}
    Mechanical and Aerospace Engineering\\
    University of California, Davis\\
    One Shields Avenue, Davis, CA, USA 95616\\
    e-mail: jkm@ucdavis.edu, mhubbard@ucdavis.edu, rahess@ucdavis.edu\\
  \end{tabular}
\end{center}

\section*{ABSTRACT}
%
We present a method to find an optimal handling bicycle using purely analytical
and numerical means. Given a linear parametrized bicycle vehicle model, a
simple manual controller, and a model-based metric that correlates with
subjective handling measures, we formulate the search for optimal geometric
parameters that give the best handling bicycle. Optimal bicycle designs for a
number of design speeds are discovered including bicycles with unintuitive
geometry, such as large negative trail. The resulting maximum handling quality
metrics for the optimal bicycle designs follow a logarithmic relationship with
respect to design speed. These optimal handling bicycles can be ridden at
speeds other than the design speed and, in general, handling difficulty
decreases with increasing speed. Finally, we show that there is little evidence
of correlation between bicycle open-loop stability and optimal handling.

\begin{keywords}
  bicycle,
  design,
  handling qualities,
  control,
  optimization
\end{keywords}

\section{INTRODUCTION}
%
Although the bicycle was invented nearly 200 years ago, by roughly 1890 its
design had evolved to the ``safety bicycle'', with similar geometry and mass
properties to those of most bikes used today. The increasing future adoption of
the bicycle as the favored vehicle for short trips depends heavily on its
rideability, that is, the degree to which the average rider is easily able to
manipulate and maneuver the vehicle to accomplish the various tasks necessary
for its use in transportation. At present, small design  modifications by
designers in the size and shape of the bike are made in an \textit{ad hoc},
trial and error, way. Although some efforts have been made recently to
understand the effects of these various design parameters on the vehicle's
passive stability (with a fixed rider) at low speeds, less is understood
regarding how these same parameters affect bicycle rideability.

The lateral handling qualities of bicycles and motorcycles are one of the most
important determinants of their rideability, safety, and usefulness. The
subject ``handling qualities'' had its birth in the assessment of aircraft. In
these applications, handling qualities referred to those qualities or
characteristics that determine the ease and precision with which a pilot may
complete a given task~\cite{Cooper1969}. Substitute ``bicycle'' or
``motorcycle'' for ``aircraft'' and ``rider'' for ``pilot'' and a workable
definition of single-track vehicle handling qualities is
evident~\cite{Hess2012}.

In this paper we combine three theoretical constructs to study the lateral
handling qualities: a) the Whipple dynamic
model~\cite{Whipple1899,Meijaard2007} of the bicycle itself, b) a rider
feedback control model that describes how the steering torques used to balance
and track desired motions are chosen, and c) a measure of the resulting
handling qualities, the ease with which this control is accomplished. We then
use computational optimization techniques to search in the vehicle design
parameter space for designs that make the handling qualities as good as
possible, given the design constraints.

A model of bicycle rider lateral balance and tracking control was presented in
Reference~\cite{Hess2012}, and included a frequency-domain based Handling
Qualities Metric (HQM) for evaluating bicycle handling qualities based upon a
similar metric used for the evaluation of aircraft handling qualities in
Reference~\cite{Hess2006}. Figures \ref{fig:inner-loops} and
\ref{fig:outer-loops} show the rider model as a five-loop feedback system where
the rider control ``dynamics'' are captured by a simple model of the rider's
neuromuscular system and five feedback ``gains.'' In the figures, $T_\delta$,
$\phi$, $\psi$ and $y_q$ represent rider steer torque input, vehicle roll,
vehicle heading, and vehicle lateral deviation, respectively using the notation
defined in Reference~\cite{Meijaard2007}. Reference~\cite{Hess2012} exercises
the model using the physical parameters of six actual bicycles, showing that
variations in both vehicle speed and physical parameters can have effects on
handling, with the former being dominant for standard bicycle designs.
%
\begin{figure}[b]
  \centering
  \includegraphics{figures/inner-loops.pdf}
  \caption{The three inner loops of the manual control system account for the
    use of vestibular, proprioceptive and visual sensory information in balance
    control.}
  \label{fig:inner-loops}
\end{figure}
%
\begin{figure}
  \centering
  \includegraphics{figures/outer-loops.pdf}
  \caption{The outer two loops rely only on visual sensing and model the
    tracking performance.}
  \label{fig:outer-loops}
\end{figure}

These observations allow us to postulate that there is a set of optimal
physical parameters that minimize the peak of the analytically predicted HQM.
The research herein details a constrained optimization approach that discovers
the optimal handling bicycle for various design speeds. The approach makes use
of the model and gain selection technique presented in
Reference~\cite{Hess2012} coupled with constraints that bound a solution space
of physically realizable bicycles. We present candidate optimal bicycle designs
and compare them to the evolutionarily derived bicycle design that we know
today.

\section{BICYCLE (PLANT) DESCRIPTION}
%
In this work, we make use of the linearized Whipple bicycle
model~\cite{Whipple1899} with parametrizations and numerical values as
presented in Reference~\cite{Meijaard2007}. This bicycle dynamic model has two
non-ignorable coordinates, roll angle $\phi$ and steer angle $\delta$, in
addition to which we add the two ignorable coordinates, heading angle $\psi$
and front wheel lateral deviation $y_q$, needed to describe tracking behavior.
The plant is manipulated with a single input, steer torque, $T_\delta$. The
plant model is parametrized by 25 geometric and inertial parameters in addition
to the nominal speed $v$ and acceleration due to gravity $g$. This model is
capable of exhibiting open-loop stability and non-minimum phase behavior, two
important dynamical properties of single track vehicles, and is thus considered
rich enough for realistic control studies. In the optimization analysis below
we focus on four geometric parameters to explore optimal bicycle designs: trail
$c$, wheelbase $w$, steer axis tilt $\lambda$, and front wheel radius $r_F$,
which are depicted in Figure~\ref{fig:bicycle-geometry}. It is important to
note that the other 21 parameters remain fixed in the analysis. For example,
the front wheel moments of inertia do not grow or shrink with changes in front
wheel radius.
%
\begin{figure}
  \centering
  \includegraphics{figures/bicycle-geometry.pdf}
  \caption{The basic geometry of the bicycle model.}
  \label{fig:bicycle-geometry}
\end{figure}

\section{CONTROLLER DESCRIPTION}
\label{sec:controller-description}
%
We make use of the manual controller proposed in Reference~\cite{Hess2012} to
simulate the behavior of a human rider. The form of this controller was chosen
to meet the following requirements:
%
\begin{enumerate}
  \item Control is compensatory in nature with no feed-forward or preview
    elements.
  \item Roll stabilization is the primary task, with path following addressed
    in the outer loops. The system should be stable in roll before closing the
    path following loops.
  \item The input to the bicycle model is steer torque, $T_\delta$ (as opposed
    to steer angle, $\delta$).
  \item The closed roll rate loop should exhibit similarities to the rate
    feedback loops found in more general laboratory tracking tasks of a human
    operator.
  \item The system should be simple. In our case, only simple gains are needed
    to stabilize the system and close all the loops.
  \item We should see evidence of the crossover model~\cite{McRuer1974} in the
    roll, heading, and lateral deviation loops.
\end{enumerate}

To achieve the above controller requirements, the five sequential loop
controller structure in Figs.~\ref{fig:inner-loops} and \ref{fig:outer-loops}
was implemented around a simple second order neuromuscular model, $G_{nm}(s)$,
and the bicycle plant, where
%
\begin{align}
  G_{nm}(s) = \frac{\omega_{nm}^2}{s^2 + 2\zeta_{nm}\omega_{nm}s + \omega_{nm}^2}
\end{align}
%
with $\omega_{nm}=30~\si{\radian\per\second}$ and $\zeta_{nm}=0.707$.

The two inner-most gains, $k_\delta$ and $k_{\dot{\phi}}$, are chosen such that
a pair of poles in each of the two first sequential closed loops lie on the
lines in the complex plane that have damping ratios specified as
$\zeta_\delta=0.1289$ and $\zeta_{\dot{\phi}}=0.0855$. There is a unique
solution, which can be found by solving the set of 12 nonlinear equations that
are formed using an Ackermann-like pole placement approach. This constraint on
the controller design ensures that the fourth controller requirement is met and
that the inner loop poles are in a location suitable for closing the outer
loops to meet the fifth requirement. The outer three gains are selected such
that the open roll loop transfer function crosses over at
2~\si{\radian\per\second} and the two outer-most open-loops cross over at 1
~\si{\radian\per\second} and 0.5 ~\si{\radian\per\second}, respectively.
Further, details on the gain selection are given in \cite{Hess2012} and
\cite{Moore2012}. It is also important to note, as will be shown in the next
section, that this controller structure has been demonstrated to realistically
predict actual human control activity in Reference~\cite{Moore2012}.

\section{HANDLING QUALITIES}
%
In general, the handling qualities of a manually controlled vehicle, be it an
aircraft, automobile or bicycle, refer to the ease and precision with which the
human can complete a given control task. The Cooper-Harper rating scale of
Figure~\ref{fig:cooper-harper-rating-scale} from Reference~\cite{Cooper1969}
has been universally adopted as a means of quantifying pilot opinion of
aircraft handling qualities. It has also been used in estimating the handling
qualities of automobiles using a model of the human driver in
Reference~\cite{Modjtahedzadeh1993}. Other rating scales have also been
utilized to delineate ease of control from the human's standpoint. For example,
the work of Reference~\cite{Weir1978} employed a composite of four rating
scales devoted to \textit{(1) Task Difficult/Ease of Accomplishment, (2)
Vehicle Handling Qualities, (3) Rider's Control Effort, and (4) Vehicle
Directional Responsiveness}. The cited study focused on motorcycle handling.
%
\begin{figure}
  \centering
  \includegraphics[width=5in]{figures/cooper-harper-rating-scale.png}
  \caption{The Cooper-Harper rating scale from Reference~\cite{Cooper1969}.}
  \label{fig:cooper-harper-rating-scale}
\end{figure}

For the purposes of the present analytical study, it was of interest to employ
a mathematical model of the human bicycle rider to predict the handling
qualities of various bicycles. Indeed, this was one of the primary
contributions of Reference~\cite{Hess2012}. A brief background on the procedure
used in Reference~\cite{Hess2012} is in order. The handling qualities
prediction scheme that was utilized borrowed heavily upon studies involving
aircraft. A useful introduction and history can be found in
Reference~\cite{Hess1990}, where a fundamental handling qualities theory
introduced by Smith in Reference~\cite{Smith1976} was explored in the context
of a specific pilot model discussed in Reference~\cite{Hess1985}.

Simply stated, the handling qualities theory just mentioned is that, in any
closed-loop tracking task, e.g., flying an aircraft, driving a car, riding a
bicycle, \textbf{rate control of any ``primary vehicle response variable'' is
of fundamental importance from the standpoint of perceived vehicle handling
qualities}.

As an example, in flying an aircraft, one such primary vehicle response
variable would be aircraft pitch attitude. Thus, pitch-rate control activity is
one determinant of handling qualities. The author of Reference~\cite{Smith1976}
further held that a physiological measure for pilot opinion ratings is the rate
at which nerve impulses (or an equivalent measure) arrive at a point in the
central nervous system where all signals due to rate control are summed or
operated upon. Furthermore, it was Smith's contention that a model for human
pilot dynamics that structurally approximates human signal processing will lead
to a natural and physical measure for pilot opinion rating. In terms of a
single-axis tracking task, the model of Figure~\ref{fig:aircraft-controller} is
offered as a simplified approximation for human signal processing in that it
explicitly incorporates the feedback of the time-rate-of-change of the primary
response variable $M$, i.e. $U_M = \frac{dM}{dt}$.
%
\begin{figure}
  \centering
  \includegraphics[width=4in]{figures/aircraft-controller.png}
  \caption{A simplified model of pilot control behavior in single-axis tracking tasks~\cite{Hess2006}.}
  \label{fig:aircraft-controller}
\end{figure}

Reflecting this thesis, the work of Reference~\cite{Hess2006} held that the
amount of ``power'' in the output rate feedback signal ($U_M$) in a model of
the human controller such as that of Figure~\ref{fig:aircraft-controller} could
be mapped into an estimate of the human's perception of handling qualities.
Rather than estimating the power in $U_M$, Reference~\cite{Hess2006} proposed a
Handling Qualities Metric (HQM) to be the magnitude of the transfer function
between the commanded input and the primary rate loop variable as a function of
frequency, normalized by $|K_p|$. Or referring to
Figure~\ref{fig:aircraft-controller},
%
\begin{equation}
  \textrm{HQM} = \left| \frac{U_M}{C}(j\omega) \right| \frac{1}{|K_p|}
  \label{eq:hqm-def}
\end{equation}

The advantage of the HQM as opposed to calculating the power in $U_M$ for a
particular task is that the HQM is readily found based upon the human/vehicle
model itself, absent the necessity of a complete simulation of the task at
hand. The $|\frac{U_M}{C}|$ reflects how the command signal $C$ produces power
in the output rate $U_M = \frac{dM}{dt}$ which the pilot uses to control $M$.
Reference~\cite{Hess2006} demonstrated that the HQM could be used to predict
Cooper-Harper ratings obtained from a well-controlled, pilot-in-the-loop
tracking study summarized in Reference~\cite{McDonnell1968} in which a variety
of controlled-element dynamics were employed and Cooper-Harper pilot opinion
ratings were recorded.  Figure~\ref{fig:hqm-various-plants} is the result.  The
horizontal bounds in Figure~\ref{fig:hqm-various-plants} were created in
Reference~\cite{Hess2006} based upon the maximum values of the HQMs and the
recorded handling qualities rating.  Here handling qualities ``levels'' are
defined using definitions associated with the Cooper-Harper scale, i.e., Level
1 means numerical Cooper-Harper ratings from 1 to 3.5, Level 2 means numerical
ratings from 3.5 to 6.5 and Level 3 means numerical ratings larger than 6.5.
Figure~\ref{fig:hqm-various-plants} shows that as reported handling qualities
improve the peak magnitude of the associated HQM decreases.
%
\begin{figure}
  \centering
  \includegraphics[width=4in]{figures/hqm-various-plants.png}
  \caption{The Handling Qualities Metrics}
  \label{fig:hqm-various-plants}
\end{figure}

The division by $|K_p|$ in Equation~\ref{eq:hqm-def} follows the procedure
established in Reference~\cite{Hess2006}. In the latter reference, the
$\frac{1}{|K_p|}$ factor was introduced to allow an effective comparison of
different controlled elements (the ``vehicle'' of
Figure~\ref{fig:aircraft-controller}) using the experimental, single-axis
tracking data of Reference~\cite{McDonnell1968}. The open-loop human/vehicle
transfer functions in these tasks exhibited variable crossover frequencies.
Dividing by $|K_p|$ in the modeling procedure effectively mitigated the effect
of this crossover frequency variation in the handling qualities prediction
scheme that was developed. It is thus essential to retain $\frac{1}{|K_p|}$
factor to utilize the handling qualities bounds shown in
Figure~\ref{fig:hqm-various-plants} for the bicycle handling study.

As described in Reference~\cite{Hess2012}, the pilot model of
Figure~\ref{fig:aircraft-controller} was modified and extended to provide a
model of the bicycle rider. The HQM concept was also extended to provide an
estimate of handling qualities for the purposes of bicycle research. Level 1
handling qualities would be described as ``little rider compensation
required.'' Level 2 would be described as ``moderate rider compensation
required.'' Level 3 would be described as ``significant rider compensation
required.'' Here, ``rider compensation'' refers to the amount of rate feedback
activity in what has been termed the ``primary control loop'' employed by the
rider to ensure stability and desired performance. In a sequential loop closure
paradigm as employed in Figs.~\ref{fig:inner-loops} and \ref{fig:outer-loops},
the primary control loop is the inner-most control loop (excluding manipulator
deflection feedback) in which the human utilizes vehicle response to generate
control inputs. Finally, it is important to emphasize that use of the bounds of
Figure~\ref{fig:hqm-various-plants} requires that the crossover frequency of
the primary control loop be 2~\si{\radian\per\second}.  This is due to the fact
that a 2~\si{\radian\per\second} crossover frequency was used in the pilot
models that created Figure~\ref{fig:hqm-various-plants}.  In terms of the
bicycle model of Figures~\ref{fig:inner-loops} and \ref{fig:outer-loops}, the
primary control loop is bicycle roll attitude $\phi$, with roll-rate providing
the rate feedback system shown in Figure~\ref{fig:aircraft-controller}.

\section{OPTIMAL HANDLING QUALITIES}
%
With the peak of the HQM definition provided in the previous section, a single
value that measures ease of handling for a bicycle model at a given speed, we
have an analytical structure that can be utilized to ask questions about the
bicycle's optimal design to minimize the peak HQM. In particular, we
investigate the design space of several geometric parameters to find the
minimal peak HQM for a bicycle at a desired design speed. We choose  to
investigate four of the 25 parameters of the benchmark Whipple
model~\cite{Meijaard2007} that are commonly associated with importance in
handling: $p=[c \quad w \quad \lambda \quad r_F]$ (see
Figure~\ref{fig:bicycle-geometry}). We include bounds on these geometric
parameters so that many physically unrealizable bicycles are eliminated from
the allowable parameter space. For simplicity we begin with only four bounds,
but in future work we plan to explore a more complete set of constraints. This
minimization can be mathematically formulated as
%
\begin{equation}
  \begin{aligned}
    & \underset{p}{\text{minimize}} & & max(\textrm{HQM}(p)) \\
    & \text{subject to} & & \\
    & & & -\infty <  c < \infty \\
    & & & \frac{m_H x_H + m_B x_B}{m_H + m_B} \leq w < \infty \\
    & & & -\pi / 2 \leq \lambda \leq \pi/2 \\
    & & & 0 < r_F < \infty
  \end{aligned}
\end{equation}
%
where the trail $c$ can take on any real value, the wheelbase $w$ is
constrained such that the center of mass of the vehicle is always between the
wheel contact points, the steer axis tilt $\lambda$ can take on any effective
angle, and the front wheel radius $r_F$ cannot be zero or negative. The
parameters $m_H, m_B$ are the masses of the front assembly and the rear frame
with rider and $x_H, x_B$ are the longitudinal locations of those two mass
centers as defined in Reference~\cite{Meijaard2007}.

It is important to note that it is not always possible to find a suitable
controller using the methods described in
Section~\ref{sec:controller-description} for every bicycle in the above defined
parameter space due to both the lack of robustness of the gain selection
algorithm and the limitations incurred from the controller being representative
of a human. To accommodate this deficiency, any set of selected gains that
create a closed loop system that is not stable is assigned an peak HQM that is
proportional to the largest real part of the unstable poles of the system with
a minimum peak HQM of 10.

Using this problem formulation we find the optimal bicycle design at a given
speed using the Covariance Matrix Adaptation Evolution
Strategy~\cite{Hansen1996}, a derivative-free stochastic optimization
algorithm. The algorithm is suited to this problem due to the non-linearity of
the objective function and the difficulty in computing its gradient. The
initial guess for the optimization runs is set to the nominal benchmark
parameter values found in Reference~\cite{Meijaard2007} and $\sigma_p$, the
initial standard deviation of the search spread, is set to $\sigma_p=[0.5 \quad
3.0 \quad 0.3\pi \quad 0.2]$.

\section{RESULTS}
%
\subsection{Example Optimal Bicycle}
%
An example optimal bicycle found for a design speed of 5.0 m/s is shown in
Figure~\ref{fig:example-optimal-bicycle}. With the optimal combination of
geometric parameters,
$c=-0.239\si{\meter},\lambda=0.678\si{\radian},r_F=0.188\si{\meter},w=1.507\si{\meter}$,
the peak of the handling quality metric can be lowered from 4.64 to 0.41. The
wheelbase is larger than the nominal benchmark bicycle, the trail is rather
largely negative, the steer axis is more laid back, and the front wheel
diameter is roughly halved. This optimal handling bicycle is not open-loop
stable at any speed from 0~\si{\meter\per\second} to 10~\si{\meter\per\second}
and the weave oscillation frequency is increased by a factor of about two. The
eigenvalues do not change as drastically with respect to speed as do those of
the benchmark model with nominal parameters.
%
\begin{figure}
  \centering
  \includegraphics{figures/example-optimal-bicycle.pdf}
  \caption{Comparison of an optimal bicycle designed for traveling at 5.0 m/s
    (black) to the benchmark bicycle design (grey) which represents a
    relatively normal bicycle. The top figure shows the differences in the
    fundamental geometry of the two bicycles. The B, H, R, and F symbols
    represent the mass centers of the rear frame and rider, the front assembly,
    the rear wheel, and the front wheel, respectively. The lower left plot
    shows how the uncontrolled plant dynamics, i.e. eigenvalues, vary with
    respect to speed for both bicycles. The lower right plot shows the handling
    quality metric for both bicycle designs at 5.0~\si{\meter\per\second}.}
  \label{fig:example-optimal-bicycle}
\end{figure}

\subsection{Optimal Bicycles}
%
Figure~\ref{fig:all-optimal-bicycles} shows optimal designs found for travel
speeds of \SIlist{2;4;6;8;10}{\meter\per\second}. The corresponding optimal
parameters for each of these designs are provided in
Table~\ref{tab:optimal-parameters}. The bicycle designs do not seem to follow a
continuous trend in change in geometry, which could possibly be due to finding
only local minima. The designs for 4~\si{\meter\per\second} and
10~\si{\meter\per\second} share similarities, for example the large front wheel
diameter and a relatively large negative trail. It is important to note that we
did not allow the inertial properties of the wheel to vary, but rather only the
geometry. The 6~\si{\meter\per\second} and 8~\si{\meter\per\second} both have
positive trail and very small front wheels. The 2~\si{\meter\per\second} design
has a very short wheelbase at the parameter boundary and large positive trail.
This design may not be physically realizable. It is important to note that
finding a closed loop stable design at 2~\si{\meter\per\second} is very
difficult due to the limitations of the controller. The minimal value of the
HQM decreases approximately logarithmically with increases in speed.
%
\begin{table}
  \caption{The optimal geometric parameters found for each design speed $v$.}
  \label{tab:optimal-parameters}
  \centering
  \begin{tabular}{llrrrrrr}
    \toprule
    $v$ & \si{\meter\per\second} & 2 & 4 & 6 & 8 & 10 \\
    $c$          & \si{\meter}  & 0.328 & -0.206 & 0.026 & 0.202 & -0.124 \\
    $w$          & \si{\meter}  & 0.327 &  1.103 & 0.817 & 0.993 &  0.876 \\
    $\lambda$    & \si{\radian} & 0.524 &  0.265 & 0.058 & 0.207 &  0.127 \\
    $r_f$        & \si{\meter}  & 0.331 &  0.477 & 0.037 & 0.080 &  0.490 \\
    \bottomrule
  \end{tabular}
\end{table}
%
\begin{figure}
  \centering
  \includegraphics{figures/all-optimal-bicycles.pdf}
  \caption{Optimal bicycle designs for design speeds of
    \SIlist{2;4;6;8;10}{\meter\per\second} along with the minimal peak HQM
    value at that speed plotted on a logarithmic scale.}
  \label{fig:all-optimal-bicycles}
\end{figure}

\section{DISCUSSION}
%
\subsection{Handling and Stability}
%
In general, other manually controlled vehicles with good performance are not
necessarily open-loop stable. In fact, some maneuverable modern fighter jets
are unstable when their human-in-the-loop control system is
removed~\cite{Wikipedia2016}. Race cars are often designed such that they are
``on the edge of stability'', meaning that without fine feedback control from
the driver in extreme maneuvers the vehicle would skid. The Wright Brothers
recognized the fact that stability can be at odds with maneuverability. They
built an unstable aircraft with emphasis on the pilot's ability to control the
vehicle and ultimately demonstrated one of the first successful powered
flights~\cite{Anderson1986}. But vehicles that are not designed for high
maneuverability are often open-loop stable permitting the human to allow the
vehicle to recover from small disturbances on its own; active control is
required only in more pressing activities. Reference~\cite{Wilson2004} argues
this view as a possible method that humans utilize to control bicycles.

The above handling quality metric reflects the level of compensation required
by the rider. With the reasoning above one might be inclined to believe a
bicycle design with optimal handling qualities would be open-loop stable at the
design speed. Yet, none of the discovered designs are stable at the design
speed. The \SIlist{6;8}{\meter\per\second} designs are marginally unstable in
the capsize mode at the design speed, whereas the rest are unstable in the
weave mode. And interestingly, the \SIlist{4;5;10}{\meter\per\second} cases
exhibit no stable speeds at all up to 10~\si{\meter\per\second}. Given the
hypothesized handling quality metric, a conclusion is that theoretically well
handling bicycles do not necessarily exhibit open-loop stability nor is there
any clear connection to the plant's stability and the handling qualities of the
vehicle.

\subsection{Speed and Handling}
%
The lack of continuity of the parameters of the optimal bicycles at adjacent
design speeds in Figure~\ref{fig:all-optimal-bicycles} indicates that the
gradient of the HQM in the search space with respect to design speed may not be
very constant or that we may be finding local optima in the search. If the
former is true, designing a single bicycle with either adjustable geometry that
is speed dependent or with a steer-by-wire system that can manipulate the
dynamics to match these designs becomes difficult. It is also not clear whether
the rider would feel comfortable with the change in lateral dynamics of a given
design as speed increased or decreased. Another optimization approach would be
to find a set of bicycles that have an equivalently low handling quality metric
for each speed, which might possibly produce more continuity between designs.
It is definitely worth investigating further the effects of speed on handling.
%
Reference~\cite{Hess2012} showed that vehicle speed, $v$, has a strong effect
on the handling quality metric, much more so than the differences in the
physical parameters of the several real bicycle designs considered. In
particular, the handling quality metric decreases dramatically with speed as
shown in Figure~\ref{fig:hqm-vs-speed}.
%
\begin{figure}
  \centering
  \includegraphics[width=4in]{figures/hqm-vs-speed.pdf}
  \caption{Peak HQM plotted against speed for the six real bicycles presented in Reference~\cite{Hess2012}.}
  \label{fig:hqm-vs-speed}
\end{figure}

This then begs the question of whether a bicycle designed to have a minimum
handling quality metric at a given speed will also have good handling qualities
when ridden at other speeds or will it exhibit this or another trend with
respect to speed? Figure~\ref{fig:hqm-vs-speed-optimal} shows the speed
dependence of the peak HQM for the five bicycle designs shown in
Figure~\ref{fig:all-optimal-bicycles}. The decrease in HQM with speed is
evident in most of the bicycles, but only the \SIlist{2;10}{\meter\per\second}
designs reflect the same smooth trends as Figure~\ref{fig:hqm-vs-speed}. The
other designs are difficult to find controllers for at all the shown speeds
using the method described in Section~\ref{sec:controller-description} and the
discovered controllers give more erratic peak values of the handling quality
metric. All the optimal designs except for the one at 2~\si{\meter\per\second}
have lower handling quality metrics than the real bicycles in
Figure~\ref{fig:hqm-vs-speed}. The 2~\si{\meter\per\second} bicycle has a peak
HQM of about 7 but at higher speeds the peak HQM is worse than the real
bicycles shown in Figure~\ref{fig:hqm-vs-speed}. This also indicates that the
handling quality metric surface in the parameter search space may have steep
frequent valleys or that it is not necessarily smooth due to the
non-linearities produced by the gain selection algorithm. Improvements to the
robustness of the controller gain selection would help answer this question.
%
\begin{figure}
  \centering
  \includegraphics[width=4in]{figures/hqm-vs-speed-optimal.pdf}
  \caption{Peak HQM plotted against speed for the five optimal bicycle designs.}
  \label{fig:hqm-vs-speed-optimal}
\end{figure}

The design speed and the speed at which the particular design is traveling have
strong influences on the handling quality metric. This speed dependence must be
taken into account when designing a bicycle for good handling. It also may not
be possible to design a bicycle that has optimal handling at every desired
travel speed.

\subsection{Unintuitive Optimal Geometry}
%
One of the advantages of approaching this problem from a optimization
perspective is that combinations of physical parameters that are less than
intuitive may be discovered. For example, the bicycle designs found for
\SIlist{4;5;10}{\meter\per\second} have substantial negative trail, which is
rarely seen in common bicycle designs. Most designs assume that a small amount
of positive trail is ideal. Only the 2~\si{\meter\per\second} design has a
significantly odd wheelbase with respect to standard designs, although some of
the optimal designs have long wheelbases which are close to that of a tandem
bicycle. Front wheel diameters ranging from 0.037\si{\meter} to
0.490~\si{\meter} were discovered. The very small diameters would likely not be
implemented on a real bicycle due to the size of obstacles one must traverse.
No negative values of steer axis tilt were discovered for optimal handling but
interestingly a very large laid back angle was found for the
5~\si{\meter\per\second} bicycle and a very steep small angle for the
6~\si{\meter\per\second} bicycle. It is not yet apparent why very different
geometries are found for nearly equal design speeds.

The discovered optimal bicycles are, in general, physically realizable bicycles
and not too wildly different than the safety bicycle design that is most
common. A broader search of the 25 plant parameters could find other unexpected
bicycle designs. There do exist various real bicycles that have odd geometry
(e.g. low-riders, recumbents, tandems, tall-bikes) and riders find them varying
in difficulty to command and maneuver. It would be interesting to learn whether
any of these existing bicycles come close to the optimal predicted designs.

% TODO : Bonus...find out what an optimal handling bicycle would look like with gravity on the moon.

\section{CONCLUSION}
%
The approach we have taken to find an optimal handling bicycle is attractive
for many reasons. The analytical nature of the presented method provides a
straightforward computation to assess handling. With the addition of
optimization this provides a bicycle designer with a tool to home in on
bicycles that give optimal lateral handling characteristics. As far as we know,
there presently do not exist any tools that give the designer the ability to
evaluate how well a given design will suit the rider.

This theoretical approach hinges on the HQM being a good predictor of the
subjective feelings of a rider. One obvious next step is to build some or all
of the bicycles that were found which would permit human riders to assess their
actual lateral handling qualities. Our theory relies heavily on the correlation
of the HQM with pilot ratings of hand manipulated SISO tracking tasks, so
further experimentation is needed to confirm this correlation for single track
vehicles. A realistic simulator or variable parameter bicycle would allow more
extensive experimental handling studies to be done, the results of which could
be compared to our HQM directly to corroborate the correlations. These might
more strongly validate the use of the handling quality metric for the case of
the bicycle.

The method can also be enhanced by increasing the dimension of the physical
parameter search space. With a larger search space and an expanded set of
important constraints that keep the bicycle design bounded to physically
realizable models, one may discover very different designs than we have
presented here. There are also details in the controller gain selection that
can be improved to increase the robustness of the numerical algorithm.

Other future work could include creating an adjustable parameter bicycle, with
consequent variable handling qualities, even allowing the physical dimensions
of the bicycle to change as speed increases and decreases. The method presented
here could help when designing the target configurations for the various
speeds. One could also create a plant model based on a fixed bicycle design
paired with a steer-by-wire controller. One could then search the controller
parameter space to give a desired handling quality metric, thus providing a
method of designing assistive controllers for single track vehicles.

We see the present work as a step forward in answering how a
manually-controlled vehicle's physical properties affect the ease of
manipulation and control for the human operator. We believe that it can provide
a practical means of assessing bicycle designs in ways that have not previously
possible.

\section{REPRODUCIBILITY}
%
All of the source code, data, and documents needed to reproduce the presented
results and this paper can be found at the repository hosted at
\url{https://github.com/moorepants/BMD2016}.

\bibliographystyle{unsrt}
\bibliography{references}

\end{document}
